\documentclass[nofonts,]{tufte-handout}

% ams
\usepackage{amssymb,amsmath}

\usepackage{ifxetex,ifluatex}
\usepackage{fixltx2e} % provides \textsubscript
\ifnum 0\ifxetex 1\fi\ifluatex 1\fi=0 % if pdftex
  \usepackage[T1]{fontenc}
  \usepackage[utf8]{inputenc}
\else % if luatex or xelatex
  \makeatletter
  \@ifpackageloaded{fontspec}{}{\usepackage{fontspec}}
  \makeatother
  \defaultfontfeatures{Ligatures=TeX,Scale=MatchLowercase}
  \makeatletter
  \@ifpackageloaded{soul}{
     \renewcommand\allcapsspacing[1]{{\addfontfeature{LetterSpace=15}#1}}
     \renewcommand\smallcapsspacing[1]{{\addfontfeature{LetterSpace=10}#1}}
   }{}
  \makeatother
\fi

% graphix
\usepackage{graphicx}
\setkeys{Gin}{width=\linewidth,totalheight=\textheight,keepaspectratio}

% booktabs
\usepackage{booktabs}

% url
\usepackage{url}

% hyperref
\usepackage{hyperref}

% units.
\usepackage{units}


\setcounter{secnumdepth}{-1}

% citations
\usepackage{natbib}
\bibliographystyle{plainnat}

%% tint override
\setcitestyle{round} 

% pandoc syntax highlighting

% longtable

% multiplecol
\usepackage{multicol}

% strikeout
\usepackage[normalem]{ulem}

% morefloats
\usepackage{morefloats}


% tightlist macro required by pandoc >= 1.14
\providecommand{\tightlist}{%
  \setlength{\itemsep}{0pt}\setlength{\parskip}{0pt}}

% title / author / date
\title{An Agent-Based Model of COVID-19 Transmission}
\author{Jonathan M. Gilligan and Kelsea B. Best}
\date{2020-05-01}

%% -- tint overrides
%% fonts, using roboto (condensed) as default
\usepackage[cmintegrals,cmbraces]{newtxmath}
\usepackage{ebgaramond-maths}
\usepackage{nimbusmononarrow}

%% colored links, setting 'borrowed' from RJournal.sty with 'Thanks, Achim!'
\RequirePackage{color}
\definecolor{link}{rgb}{0.1,0.1,0.8} %% blue with some grey
\hypersetup{
  colorlinks,%
  citecolor=link,%
  filecolor=link,%
  linkcolor=link,%
  urlcolor=link
}

%% macros
\makeatletter

%% -- tint does not use italics or allcaps in title
\renewcommand{\maketitle}{%     
  \newpage
  \global\@topnum\z@% prevent floats from being placed at the top of the page
  \begingroup
    \setlength{\parindent}{0pt}%
    \setlength{\parskip}{4pt}%
    \let\@@title\@empty
    \let\@@author\@empty
    \let\@@date\@empty
    \ifthenelse{\boolean{@tufte@sfsidenotes}}{%
      %\gdef\@@title{\sffamily\LARGE\allcaps{\@title}\par}%
      %\gdef\@@author{\sffamily\Large\allcaps{\@author}\par}%
      %\gdef\@@date{\sffamily\Large\allcaps{\@date}\par}%
      \gdef\@@title{\begingroup\fontseries{b}\selectfont\LARGE{\@title}\par}%
      \gdef\@@author{\begingroup\fontseries{l}\selectfont\Large{\@author}\par}%
      \gdef\@@date{\begingroup\fontseries{l}\selectfont\Large{\@date}\par}%
    }{%
      %\gdef\@@title{\LARGE\itshape\@title\par}%
      %\gdef\@@author{\Large\itshape\@author\par}%
      %\gdef\@@date{\Large\itshape\@date\par}%
      \gdef\@@title{\begingroup\fontseries{b}\selectfont\LARGE\@title\par\endgroup}%
      \gdef\@@author{\begingroup\fontseries{l}\selectfont\Large\@author\par\endgroup}%
      \gdef\@@date{\begingroup\fontseries{l}\selectfont\Large\@date\par\endgroup}%
    }%
    \@@title
    \@@author
    \@@date
  \endgroup
  \thispagestyle{plain}% suppress the running head
  \tuftebreak% add some space before the text begins
  \@afterindentfalse\@afterheading% suppress indentation of the next paragraph
}

%% -- tint does not use italics or allcaps in section/subsection/paragraph
\titleformat{\section}%
  [hang]% shape
  %{\normalfont\Large\itshape}% format applied to label+text
  {\fontseries{b}\selectfont\Large}% format applied to label+text
  {\thesection}% label
  {1em}% horizontal separation between label and title body
  {}% before the title body
  []% after the title body

\titleformat{\subsection}%
  [hang]% shape
  %{\normalfont\large\itshape}% format applied to label+text
  {\fontseries{m}\selectfont\large}% format applied to label+text
  {\thesubsection}% label
  {1em}% horizontal separation between label and title body
  {}% before the title body
  []% after the title body

\titleformat{\paragraph}%
  [runin]% shape
  %{\normalfont\itshape}% format applied to label+text
  {\fontseries{l}\selectfont}% format applied to label+text
  {\theparagraph}% label
  {1em}% horizontal separation between label and title body
  {}% before the title body
  []% after the title body

%% -- tint does not use italics here either
% Formatting for main TOC (printed in front matter)
% {section} [left] {above} {before w/label} {before w/o label} {filler + page} [after]
\ifthenelse{\boolean{@tufte@toc}}{%
  \titlecontents{part}% FIXME
    [0em] % distance from left margin
    %{\vspace{1.5\baselineskip}\begin{fullwidth}\LARGE\rmfamily\itshape} % above (global formatting of entry)
    {\vspace{1.5\baselineskip}\begin{fullwidth}\fontseries{m}\selectfont\LARGE} % above (global formatting of entry)
    {\contentslabel{2em}} % before w/label (label = ``II'')
    {} % before w/o label
    {\rmfamily\upshape\qquad\thecontentspage} % filler + page (leaders and page num)
    [\end{fullwidth}] % after
  \titlecontents{chapter}%
    [0em] % distance from left margin
    %{\vspace{1.5\baselineskip}\begin{fullwidth}\LARGE\rmfamily\itshape} % above (global formatting of entry)
    {\vspace{1.5\baselineskip}\begin{fullwidth}\fontseries{m}\selectfont\LARGE} % above (global formatting of entry)
    {\hspace*{0em}\contentslabel{2em}} % before w/label (label = ``2'')
    {\hspace*{0em}} % before w/o label
    %{\rmfamily\upshape\qquad\thecontentspage} % filler + page (leaders and page num)
    {\upshape\qquad\thecontentspage} % filler + page (leaders and page num)
    [\end{fullwidth}] % after
  \titlecontents{section}% FIXME
    [0em] % distance from left margin
    %{\vspace{0\baselineskip}\begin{fullwidth}\Large\rmfamily\itshape} % above (global formatting of entry)
    {\vspace{0\baselineskip}\begin{fullwidth}\fontseries{m}\selectfont\Large} % above (global formatting of entry)
    {\hspace*{2em}\contentslabel{2em}} % before w/label (label = ``2.6'')
    {\hspace*{2em}} % before w/o label
    %{\rmfamily\upshape\qquad\thecontentspage} % filler + page (leaders and page num)
    {\upshape\qquad\thecontentspage} % filler + page (leaders and page num)
    [\end{fullwidth}] % after
  \titlecontents{subsection}% FIXME
    [0em] % distance from left margin
    %{\vspace{0\baselineskip}\begin{fullwidth}\large\rmfamily\itshape} % above (global formatting of entry)
    {\vspace{0\baselineskip}\begin{fullwidth}\fontseries{m}\selectfont\large} % above (global formatting of entry)
    {\hspace*{4em}\contentslabel{4em}} % before w/label (label = ``2.6.1'')
    {\hspace*{4em}} % before w/o label
    %{\rmfamily\upshape\qquad\thecontentspage} % filler + page (leaders and page num)
    {\upshape\qquad\thecontentspage} % filler + page (leaders and page num)
    [\end{fullwidth}] % after
  \titlecontents{paragraph}% FIXME
    [0em] % distance from left margin
    %{\vspace{0\baselineskip}\begin{fullwidth}\normalsize\rmfamily\itshape} % above (global formatting of entry)
    {\vspace{0\baselineskip}\begin{fullwidth}\fontseries{m}\selectfont\normalsize\rmfamily} % above (global formatting of entry)
    {\hspace*{6em}\contentslabel{2em}} % before w/label (label = ``2.6.0.0.1'')
    {\hspace*{6em}} % before w/o label
    %{\rmfamily\upshape\qquad\thecontentspage} % filler + page (leaders and page num)
    {\upshape\qquad\thecontentspage} % filler + page (leaders and page num)
    [\end{fullwidth}] % after
}{}

  
\makeatother



\begin{document}

\maketitle




\hypertarget{introduction}{%
\section{Introduction}\label{introduction}}

This model is a simple agent-based model of COVID-19 transmission in a
population connected by social networks. The transmission of COVID-19 is
modeled using a compartmental SEIR model in which individual agents
progress through disease stages of susceptible, exposed, infected or
recovered. Agents are connected to one another through non-spatial
social networks which are modeled as Watts-Strogatz small world networks
\citep{watts:collective.dynamics:1998} or Barab'asi-Albert
scale-invariant networks
\citep{albert:statistical.mechanics:2002, barabasi:emergence.scaling:1999}.
The disease may be transmitted through these network links.

\hypertarget{model-overview}{%
\section{Model Overview}\label{model-overview}}

\hypertarget{purpose}{%
\subsection{Purpose}\label{purpose}}

The purpose of this model is to use agent-based modeling to simulate the
spread of COVID-19 in communities. The model is used to simulate the
effects of different types of network connectivity (including changes to
the network connections through interventions such as ``social
isolation'') on the spread of infections.

The model does not attempt to provide reliable predictions of the future
spread of disease, but focuses on making comparisons between different
patterns of social network connections.

\hypertarget{entities-state-variables-and-scales}{%
\subsection{Entities, State Variables, and
Scales}\label{entities-state-variables-and-scales}}

There are two kinds of entities in this model: \textbf{agents} and
\textbf{links}.

\hypertarget{agents}{%
\subsubsection{Agents}\label{agents}}

\textbf{Agents} represent individual people. Each agent is initialized
with the following characteristics:

\begin{itemize}
\tightlist
\item
  \emph{age}, which is represented as both a number and as a bracket.
\item
  \emph{sex}, a binary variable representing male or female sex
  phenotype. This model does not currently treat intersex individuals.
\item
  \emph{comorbidity}, a binary variable that represents whether the
  indivisual has a comorbidity that could complicate susceptibility to
  new infections.
\item
  \emph{symptomatic}, a binary variable that controls whether the
  individaul will be symptomatic if they are infected. This is
  important, because symptomatic individuals are reported to shed
  greater amounts of virus than asymptomatic ones. Future variations on
  this model may also account for behavioral differences (symptomatic
  individuals may be more likely to self-quarantine or seek medical
  attention, whereas asymptomatic ones may continue to interact with
  others and thus spread the disease)
\item
  \emph{health status}, an ordinal factor that represents the status of
  the individual within a four-compartment \emph{SEIR} model, where
  \emph{S} represents \emph{susceptible} individuals, \emph{E}
  represents \emph{exposed} individuals who are incubating an infection
  but are not yet infectious, \emph{I} represents \emph{infectious}
  individuals who can spread the disease, and \emph{R} represents
  \emph{recovered} individuals who are no longer contagious and who have
  acquired immunity against being reinfected.
\item
  \emph{ticks}, an integer representing the number of time steps since
  the individual entered the current \emph{health status}. This affects
  the progression from \emph{E} to \emph{I} and from \emph{I} to
  \emph{R}.
\item
  \emph{shedding factor}, a real number (positive or negative) that
  represents the relative degree of viral shedding (e.g., symptomatic
  individuals are reported to have higher shedding factors than
  asymptomatic ones).
\item
  \emph{susceptibility factor}, a real number (positive or negative)
  that represents the relative susceptibility of an individual to
  contracting an infection when exposed to the virus.
\item
  \emph{EI scale}, \emph{EI shape}, \emph{IR scale}, and \emph{IR
  shape}, real numbers that parameterize progress of the disease. See
  further details under \emph{Submodels}
\end{itemize}

\hypertarget{links}{%
\subsubsection{Links}\label{links}}

*Links** represent social connections. These can be \emph{household}
connections (people who live together), \emph{social} connections of
people who live in different households, but are connected through
friendship, church, community groups, etc., and see one another
regularly, and \emph{work} connections.

Links have parameters:

\begin{itemize}
\tightlist
\item
  \emph{contact frequency}, represents the mean frequency of this kind
  of contact, in number of meetings per time step.
\item
  \emph{contact intensity}, represents the intensity of close
  interaction (e.g., visits to a health-care professional will generally
  be more intense than interactions with cashiers at a store)
\end{itemize}

The characteristics of the network is described in greater detail under
\emph{collectives}, below.

\hypertarget{spatiotemporal-scales}{%
\subsubsection{Spatiotemporal scales}\label{spatiotemporal-scales}}

This model does not explicitly represent space, and agents interact
through social networks rather than spatial proximity.

The time step represents one day. The number of ticks in a model run can
be specified by the user.

\hypertarget{environment}{%
\subsubsection{Environment}\label{environment}}

Initially, we will not model the environment. We will have agents
connected by social networks with no representation of the physical
environments they inhabit. Later versions will incorporate
infrastructure to represent connections via transportation, and spatial
distributions of housing units and workplaces.

\hypertarget{process-overview-and-scheduling}{%
\subsection{Process Overview and
Scheduling}\label{process-overview-and-scheduling}}

At each time step, two things happen:

\begin{enumerate}
\def\labelenumi{\arabic{enumi}.}
\tightlist
\item
  Each agent in the \emph{E} or \emph{I} state progresses stochastically
  toward the next state (\emph{I} or \emph{R}, respectively). The
  probability of progression may depend on the number of time steps the
  agent has been in this status. See the \emph{disease progression}
  submodel for details.
\item
  Every agent in the \emph{I} state stochastically infects its immediate
  neighbors on the network (those with whom it shares a link) who are in
  the \emph{S} status. See the \emph{infect} submodel for details.
\end{enumerate}

\hypertarget{design-concepts}{%
\section{Design Concepts}\label{design-concepts}}

\hypertarget{basic-principles}{%
\subsection{Basic Principles}\label{basic-principles}}

The transmission of COVID-19 is represented using a four-compartment
SEIR model. The SEIR model categorizes agents into \emph{susceptible},
\emph{exposed}, \emph{infectious} or \emph{recovered}.

Infection (transition from \emph{susceptible} to \emph{exposed} status)
occurs stochastically. At each time step, every \emph{susceptible} agent
that is connected to an \emph{infectious} agent by a network link has a
probability of transitioning to an \emph{exposed} state. The probability
depends on the characteristics of the two agents and of the network
link.

Agents are heterogeneous, so two agents with the same age, sex, etc. can
have different shedding and susceptibility factors. These factors are
drawn at random from probability distributions that are parameterized by
the agent characteristics (age, sex, comorbidities, and symptomatic
status)

\emph{Exposed} and \emph{infectious} agents stochastically transition to
the next stage (\emph{infectious} and \emph{recovered}, respectively),
with time-dependent probabilities that follow gamma or Weibull
distributions. The parameters of these distributions may depend on the
agent's age, sex, comorbidities, etc.

\hypertarget{emergence}{%
\subsection{Emergence}\label{emergence}}

The spread of the disease emerges from individual interactions on the
network.

\hypertarget{adaptation}{%
\subsection{Adaptation}\label{adaptation}}

Currently, the agents do not adapt their behavior to changing
conditions. Future versions may allow agents to change their social
interactions when they get sick or in response to public policies, such
as stay-at-home orders.

\hypertarget{objectives}{%
\subsection{Objectives}\label{objectives}}

The agents do not pursue objectives.

\hypertarget{learning}{%
\subsection{Learning}\label{learning}}

The agents do not learn.

\hypertarget{prediction}{%
\subsection{Prediction}\label{prediction}}

The agents do not engage in prediction.

\hypertarget{sensing}{%
\subsection{Sensing}\label{sensing}}

The agents do not currently use sensing. In the future, they may sense
aspects of their own or other agents' health.

\hypertarget{interaction}{%
\subsection{Interaction}\label{interaction}}

Agents interact through links. These are how infections are transmitted.

\hypertarget{stochasticity}{%
\subsection{Stochasticity}\label{stochasticity}}

Agent initialization is stochastic with characteristics (age, sex,
comorbidities, future symptomatic response to infection, shedding
intensity, and susceptibility) drawn from distributions that can be
specified at run-time.

Disease progress (\emph{E} to \emph{I} and \emph{I} to \emph{R}) are
stochastic, with probabilities that vary with the amount of time an
agent has been in that status.

Disease transmission is stochastic. Disease is transmitted across links
that connect \emph{infectious} to \emph{susceptible} agents. The
probability of transmission depends on the \emph{shedding intensity} of
the \emph{infectious} agent, the \emph{susceptibility} of the
\emph{susceptible} agent, and the contact characteristics of the link.

\hypertarget{collectives}{%
\subsection{Collectives}\label{collectives}}

Multiple overlapping social networks (household, social, and work)
connect agents. These networks can have different topologies that are
specified at runtime when the agents are initialized.

Currently available topologies are Strogatz-Watts \emph{small-world}
\citep{watts:collective.dynamics:1998} and Barabasi Albert
\emph{preferential attachment}
\citep{albert:statistical.mechanics:2002, barabasi:emergence.scaling:1999}.
The big difference between these is that the degree of connection is
fairly uniformly distributed in the Strogatz Watts model, but is very
unequally distributed in Barab'asi-Albert networks, with a few highly
connected nodes that may be able to simulate super-spreaders.

\hypertarget{observation}{%
\subsection{Observation}\label{observation}}

At each time step we record the number of agents in each health status
(\emph{S}, \emph{E}, \emph{I}, or \emph{R}).

\hypertarget{details}{%
\section{Details}\label{details}}

\hypertarget{initialization}{%
\subsection{Initialization}\label{initialization}}

\hypertarget{input-data}{%
\subsection{Input Data}\label{input-data}}

\hypertarget{submodels}{%
\subsection{Submodels}\label{submodels}}

\renewcommand\refname{References}
\bibliography{covid-abm.bib}



\end{document}
